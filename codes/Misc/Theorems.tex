\subsubsection{Kirchhoff's Theorem}
Denote $L$ be a $n \times n$ matrix as the Laplacian matrix of graph $G$, where $L_{ii} = d(i)$, $L_{ij} = -c$ where $c$ is the number of edge $(i, j)$ in $G$.
\begin{itemize}
    \item The number of undirected spanning in $G$ is $\lvert \det(\tilde{L}_{11}) \rvert$.
    \item The number of directed spanning tree rooted at $r$ in $G$ is $\lvert \det(\tilde{L}_{rr}) \rvert$.
\end{itemize}

\subsubsection{Tutte's Matrix}
Let $D$ be a $n \times n$ matrix, where $d_{ij} = x_{ij}$ ($x_{ij}$ is chosen uniform randomly) if $i < j$ and $(i, j) \in E$, otherwise $d_{ij} = -d_{ji}$. $\frac{rank(D)}{2}$ is the maximum matching on $G$.

\subsubsection{Cayley's Formula}
\begin{itemize}
  \item Given a degree sequence $d_1, d_2, \ldots, d_n$ for each labeled vertices, there're $\frac{(n - 2)!}{(d_1 - 1)!(d_2 - 1)!\cdots(d_n - 1)!}$ spanning trees.
  \item Let $T_{n, k}$ be the number of labeled forests on $n$ vertices with $k$ components, such that vertex $1, 2, \ldots, k$ belong to different components. Then $T_{n, k} = kn^{n - k - 1}$.
\end{itemize}

\subsubsection{Erdős–Gallai theorem}
A sequence of non-negative integers $d_1 \geq d_2 \geq \ldots \geq d_n$ can be represented as the degree sequence of a finite simple graph on $n$ vertices if and only if $d_1 + d_2 + \ldots + d_n$ is even and
$$ \sum_{i = 1}^{k}d_i \leq k(k - 1) + \sum_{i = k + 1}^{n}\min(d_i, k) $$
holds for all $1 \leq k \leq n$.

\subsubsection{Havel–Hakimi algorithm}
find the vertex who has greatest degree unused, connect it with other greatest vertex.

\subsubsection{Hall's marriage theorem}
Let $G$ be a finite bipartite graph with bipartite sets $X$ and $Y$.
For a subset $W$ of $X$, let $N_G(W)$ denote the set of all vertices in $Y$ adjacent to some element of $W$. Then there is an $X$-saturating matching iff $\forall W \subseteq X, |W| \leq |N_G(W)|$

\subsubsection{Euler's planar graph formula}
$V-E+F=C+1$, $E\leq 3V-6$(?)

\subsubsection{Pick's theorem}
For simple polygon, when points are all integer, we have $A=\text{\#\{lattice points in the interior\}} + \frac{\text{\#\{lattice points on the boundary\}}}{2} - 1$