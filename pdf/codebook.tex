%!TEX program = xelatex
\documentclass[a4paper,10pt,twocolumn,oneside]{article}
\setlength{\columnsep}{10pt}                                                                    %兩欄模式的間距
\setlength{\columnseprule}{0pt}                                                                %兩欄模式間格線粗細

\usepackage{amsthm}								%定義,例題
\usepackage{amssymb}
%\usepackage[margin=2cm]{geometry}
\usepackage{fontspec}								%設定字體
\usepackage{color}
\usepackage[x11names]{xcolor}
\usepackage{listings}								%顯示code用的
%\usepackage[Glenn]{fncychap}						%排版,頁面模板
\usepackage{fancyhdr}								%設定頁首頁尾
\usepackage{graphicx}								%Graphic
\usepackage{enumerate}
\usepackage{titlesec}
\usepackage{amsmath}
\usepackage[CheckSingle, CJKmath]{xeCJK}
% \usepackage{CJKulem}

%\usepackage[T1]{fontenc}
\usepackage{courier}
\topmargin=0pt
\headsep=5pt
\textheight=780pt
\footskip=0pt
\voffset=-40pt
\textwidth=545pt
\marginparsep=0pt
\marginparwidth=0pt
\marginparpush=0pt
\oddsidemargin=0pt
\evensidemargin=0pt
\hoffset=-42pt

%\renewcommand\listfigurename{圖目錄}
%\renewcommand\listtablename{表目錄} 

%%%%%%%%%%%%%%%%%%%%%%%%%%%%%

\setmainfont{Consolas}
\setmonofont{Consolas}
\setCJKmainfont{NotoSansCJKtc-Medium}           %中文字型
\XeTeXlinebreaklocale "zh"						%中文自動換行
\XeTeXlinebreakskip = 0pt plus 1pt				%設定段落之間的距離
\setcounter{secnumdepth}{3}						%目錄顯示第三層

%%%%%%%%%%%%%%%%%%%%%%%%%%%%%
\makeatletter
\lst@CCPutMacro\lst@ProcessOther {"2D}{\lst@ttfamily{-{}}{-{}}}
\@empty\z@\@empty
\makeatother
\lstset{											% Code顯示
language=C++,										% the language of the code
basicstyle=\footnotesize\ttfamily, 						% the size of the fonts that are used for the code
%numbers=left,										% where to put the line-numbers
numberstyle=\footnotesize,						% the size of the fonts that are used for the line-numbers
stepnumber=1,										% the step between two line-numbers. If it's 1, each line  will be numbered
numbersep=5pt,										% how far the line-numbers are from the code
backgroundcolor=\color{white},					% choose the background color. You must add \usepackage{color}
showspaces=false,									% show spaces adding particular underscores
showstringspaces=false,							% underline spaces within strings
showtabs=false,									% show tabs within strings adding particular underscores
frame=false,											% adds a frame around the code
tabsize=2,											% sets default tabsize to 2 spaces
captionpos=b,										% sets the caption-position to bottom
breaklines=true,									% sets automatic line breaking
breakatwhitespace=false,							% sets if automatic breaks should only happen at whitespace
escapeinside={\%*}{*)},							% if you want to add a comment within your code
morekeywords={*},									% if you want to add more keywords to the set
keywordstyle=\bfseries\color{Blue1},
commentstyle=\itshape\color{Red4},
stringstyle=\itshape\color{Green4},
}

%%%%%%%%%%%%%%%%%%%%%%%%%%%%%

\def\footnotesize{\fontsize{8}{9}\selectfont}

\begin{document}
\pagestyle{fancy}
\fancyfoot{}
\fancyhead[L]{National Taiwan University - NTU T1}
\fancyhead[R]{\thepage}
\renewcommand{\headrulewidth}{0.4pt}
\renewcommand{\contentsname}{Contents} 

\scriptsize
\tableofcontents
%%%%%%%%%%%%%%%%%%%%%%%%%%%%%

\section{Basic}
\subsection{vimrc}
\lstinputlisting{../codes/Basic/.vimrc}
\subsection{IncreaseStackSize}
\lstinputlisting{../codes/Basic/IncStack.cpp}
\subsection{Pragma optimization}
\lstinputlisting{../codes/Basic/Pragma.cpp}
\subsection{Debugger}
\lstinputlisting[language=python]{../codes/Basic/debug_checker.py}
\subsection{Quick Random}
\lstinputlisting{../codes/Basic/random_generator.cpp}
\subsection{IO Optimization}
\lstinputlisting{../codes/Basic/IOOptimize.cpp}

\section{Data Structure}
\subsection{Bigint}
\lstinputlisting{../codes/Data_Structure/Bigint.cpp}
\subsection{Dark Magic}
\lstinputlisting{../codes/Data_Structure/ext_pbds.cpp}
\subsection{SkewHeap}
\lstinputlisting{../codes/Data_Structure/SkewHeap.cpp}
\subsection{Disjoint Set}
\lstinputlisting{../codes/Data_Structure/DisjointSet.cpp}
\subsection{Link-Cut Tree}
\lstinputlisting{../codes/Data_Structure/LCT.cpp}
\subsection{LiChao Segment Tree}
\lstinputlisting{../codes/Data_Structure/LiChao.cpp}
\subsection{Treap}
\lstinputlisting{../codes/Data_Structure/Treap.cpp}
\subsection{SparseTable}
\lstinputlisting{../codes/Data_Structure/SparseTable.cpp}
\subsection{Linear Basis}
\lstinputlisting{../codes/Data_Structure/LinearBasis.cpp}

\section{Graph}
\subsection{Euler Circuit}
\lstinputlisting{../codes/Graph/EulerCircuit.cpp}
\subsection{BCC Edge}
\lstinputlisting{../codes/Graph/BCC_Edge.cpp}
\subsection{BCC Vertex}
\lstinputlisting{../codes/Graph/BCC_Vertex.cpp}
\subsection{Bipartite Matching}
\lstinputlisting{../codes/Graph/BI_Pair.cpp}
\subsection{Minimum Cost Maximum Flow}
\lstinputlisting{../codes/Graph/CostFlow.cpp}
\subsection{General Graph Matching}
\lstinputlisting{../codes/Graph/GeneralMatching.cpp}
\subsection{Dinic}
\lstinputlisting{../codes/Graph/Dinic.cpp}
\subsection{Kuhn Munkres}
\lstinputlisting{../codes/Graph/KM.cpp}
\subsection{Flow Models}
\begin{itemize}
    \item Maximum/Minimum flow with lower/upper bound from $s$ to $t$
    \begin{enumerate}
        \item Construct super source $S$ and sink $T$
        \item For each edge $(x, y, l, u)$, connect $x \rightarrow y$ with capacity $u - l$
        \item For each vertex $v$, denote $in(v)$ as the difference between the sum of incoming lower bounds and the sum of outgoing lower bounds
        \item If $in(v) > 0$, connect $S \rightarrow v$ with capacity $in(v)$, otherwise, connect $v \rightarrow T$ with capacity $-in(v)$
        \begin{itemize}
            \item To maximize, connect $t \rightarrow s$ with capacity $\infty$, and let $f$ be the maximum flow from $S$ to $T$. If $f \neq \sum_{v \in V, in(v) > 0}{in(v)}$, there's no solution. Otherwise, the maximum flow from $s$ to $t$ is the answer.
            \item To minimize, let $f$ be the maximum flow from $S$ to $T$. Connect $t \rightarrow s$ with capacity $\infty$ and let the flow from $S$ to $T$ be $f^\prime$. If $f + f^\prime \neq \sum_{v \in V, in(v) > 0}{in(v)}$, there's no solution. Otherwise, $f^\prime$ is the answer.
        \end{itemize}
        \item The solution of each edge $e$ is $l_e + f_e$, where $f_e$ corresponds to the flow on the graph
    \end{enumerate}
    \item Construct minimum vertex cover from maximum matching $M$ on bipartite graph $(X, Y)$
    \begin{enumerate}
        \item Redirect every edge ($y \rightarrow x$ if $(x, y) \in M$, $x \rightarrow y$ otherwise)
        \item DFS from unmatched vertices in $X$ 
        \item $x \in X$ is chosen iff $x$ is unvisited
        \item $y \in Y$ is chosen iff $y$ is visited
    \end{enumerate}
    \item Minimum cost cyclic flow
    \begin{enumerate}
        \item Consruct super source $S$ and sink $T$
        \item For each edge $(x, y, c)$, connect $x \rightarrow y$ with $(cost, cap) = (c, 1)$ if $c > 0$, otherwise connect $y \rightarrow x$ with $(cost, cap) = (-c, 1)$
        \item For each edge with $c < 0$, sum these cost as $K$, then increase $d(y)$ by 1, decrease $d(x)$ by 1
        \item For each vertex $v$ with $d(v) > 0$, connect $S \rightarrow v$ with $(cost, cap) = (0, d(v))$
        \item For each vertex $v$ with $d(v) < 0$, connect $v \rightarrow T$ with $(cost, cap) = (0, -d(v))$
        \item Flow from $S$ to $T$, the answer is the cost of the flow $C + K$
    \end{enumerate}
    \item Maximum density induced subgraph
    \begin{enumerate}
        \item Binary search on answer, suppose we're checking answer $T$
        \item Construct a max flow model, let $K$ be the sum of all weights
        \item Connect source $s \rightarrow v$, $v \in G$ with capacity $K$
        \item For each edge $(u, v, w)$ in $G$, connect $u \rightarrow v$ and $v \rightarrow u$ with capacity $w$
        \item For $v \in G$, connect it with sink $v \rightarrow t$ with capacity $K + 2T - (\sum_{e \in E(v)}{w(e)}) - 2w(v)$
        \item $T$ is a valid answer if the maximum flow $f < K \lvert V \rvert$
    \end{enumerate}
\end{itemize}

\subsection{2-SAT (SCC)}
\lstinputlisting{../codes/Graph/2SAT.cpp}
\subsection{Lowbit Decomposition}
\lstinputlisting{../codes/Graph/LowbitDecomposition.cpp}
\subsection{MaxClique}
\lstinputlisting{../codes/Graph/MaxClique.cpp}
\subsection{Min-Cut}
\lstinputlisting{../codes/Graph/SW.cpp}
\subsection{Virtural Tree}
\lstinputlisting{../codes/Graph/virtual_tree.cpp}
\subsection{Tree Hashing}
\lstinputlisting{../codes/Graph/TreeHash.cpp}

\section{Math}
\subsection{Prime Table}
% \normalsize
$ 1002939109, 1020288887, 1028798297, 1038684299, $
$ 1041211027, 1051762951, 1058585963, 1063020809, $
$ 1147930723, 1172520109, 1183835981, 1187659051, $
$ 1241251303, 1247184097, 1255940849, 1272759031, $
$ 1287027493, 1288511629, 1294632499, 1312650799, $
$ 1868732623, 1884198443, 1884616807, 1885059541, $
$ 1909942399, 1914471137, 1923951707, 1925453197, $
$ 1979612177, 1980446837, 1989761941, 2007826547, $
$ 2008033571, 2011186739, 2039465081, 2039728567, $
$ 2093735719, 2116097521, 2123852629, 2140170259, $
$ 3148478261, 3153064147, 3176351071, 3187523093, $
$ 3196772239, 3201312913, 3203063977, 3204840059, $
$ 3210224309, 3213032591, 3217689851, 3218469083, $
$ 3219857533, 3231880427, 3235951699, 3273767923, $
$ 3276188869, 3277183181, 3282463507, 3285553889, $
$ 3319309027, 3327005333, 3327574903, 3341387953, $
$ 3373293941, 3380077549, 3380892997, 3381118801  $
\subsection{$\lfloor \frac{n}{i} \rfloor$ Enumeration}
$ T_0 = 1, T_{i+1} = \lfloor \frac{n}{ \lfloor \frac{n}{T_i + 1} \rfloor } \rfloor $ 
\subsection{ax+by=gcd}
\lstinputlisting{../codes/Math/ExtendedGCD.cpp}
\subsection{Pollard Rho}
\lstinputlisting{../codes/Math/PollardRho.cpp}
\subsection{Pi Count (Linear Sieve)}
\lstinputlisting{../codes/Math/MeisselLehmer.cpp}
\subsection{Range Sieve}
\lstinputlisting{../codes/Math/RangeSieve.cpp}
\subsection{Miller Rabin}
\lstinputlisting{../codes/Math/miller_rabin.cpp}
\subsection{Inverse Element}
\lstinputlisting{../codes/Math/Inverse.cpp}
\subsection{Euler Phi Function}
\lstinputlisting{../codes/Math/euler_phi.cpp}
\subsection{Gauss Elimination}
\lstinputlisting{../codes/Math/GaussElimination.cpp}
\subsection{Fast Fourier Transform}
\lstinputlisting{../codes/Math/FFT.cpp}
\subsection{High Speed Linear Recurrence}
\lstinputlisting{../codes/Math/hdlr.cpp}
\subsection{Chinese Remainder}
\lstinputlisting{../codes/Math/ChineseRemainder.cpp}
\subsection{Berlekamp Massey}
\lstinputlisting{../codes/Math/Berlekamp-Massey.cpp}
\subsection{NTT}
\lstinputlisting{../codes/Math/NTT_eddy.cpp}
\subsection{Polynomial Sqrt}
\lstinputlisting{../codes/Math/PolySqrt.cpp}
\subsection{Polynomial Division}
\lstinputlisting{../codes/Math/PolyDiv.cpp}
\subsection{FWT}
\lstinputlisting{../codes/Math/FWT.cpp}
\subsection{DiscreteLog}
\lstinputlisting{../codes/Math/DiscreteLog.cpp}
\subsection{Quadratic residue}
\lstinputlisting{../codes/Math/QuadraticResidue.cpp}
\subsection{De-Bruijn}
\lstinputlisting{../codes/Math/debruijn.cpp}
\subsection{Simplex Construction}
Standard form: maximize $\sum_{1 \leq i \leq n}{c_ix_i}$ such that for all $1 \leq j \leq m$, $\sum_{1 \leq i \leq n}{A_{ji}x_i} \leq b_j$.and $x_i \geq 0$ for all $1 \leq i \leq n$.

\begin{enumerate}
    \item In case of minimization, let $c^\prime_i = -c_i$
    \item $\sum_{1 \leq i \leq n}{A_{ji}x_i} \geq b_j \rightarrow \sum_{1 \leq i \leq n}{-A_{ji}x_i} \leq -b_j$
    \item $\sum_{1 \leq i \leq n}{A_{ji}x_i} = b_j$ 
        \begin{itemize}
            \item $\sum_{1 \leq i \leq n}{A_{ji}x_i} \leq b_j$
            \item $\sum_{1 \leq i \leq n}{A_{ji}x_i} \geq b_j$
        \end{itemize}
    \item If $x_i$ has no lower bound, replace $x_i$ with $x_i - x_i^\prime$
\end{enumerate}
\subsection{Simplex}
\lstinputlisting{../codes/Math/simplex.cpp}

\section{Geometry}
\subsection{Point Class}
\lstinputlisting{../codes/Geometry/point_class.cpp}
\subsection{Circle Class}
\lstinputlisting{../codes/Geometry/Circle.cpp}
\subsection{Line Class}
\lstinputlisting{../codes/Geometry/line_class.cpp}
\subsection{Triangle Circumcentre}
\lstinputlisting{../codes/Geometry/outer_point.cpp}
\subsection{2D Convex Hull}
\lstinputlisting{../codes/Geometry/2D_ConvexHull.cpp}
\subsection{2D Farthest Pair}
\lstinputlisting{../codes/Geometry/2D_FarthestPair.cpp}
\subsection{2D Closest Pair}
\lstinputlisting{../codes/Geometry/2D_ClosestPair.cpp}
\subsection{SimulateAnnealing}
\lstinputlisting{../codes/Geometry/SimulateAnnealing.cpp}
\subsection{Ternary Search on Integer}
\lstinputlisting{../codes/Geometry/Ternary_Search.cpp}
\subsection{Minimum Covering Circle}
\lstinputlisting{../codes/Geometry/MinCircleCover.cpp}
\subsection{KDTree (Nearest Point)}
\lstinputlisting{../codes/Geometry/KD_Tree.cpp}

\section{Stringology}
\subsection{Hash}
\lstinputlisting{../codes/String/Hash.cpp}
\subsection{Suffix Array}
\lstinputlisting{../codes/String/Suffix_Array.cpp}
\subsection{Aho-Corasick Algorithm}
\lstinputlisting{../codes/String/Aho_Corasick.cpp}
\subsection{Suffix Automaton}
\lstinputlisting{../codes/String/SAM.cpp}
\subsection{KMP}
\lstinputlisting{../codes/String/KMP.cpp}
\subsection{Z value}
\lstinputlisting{../codes/String/Z_Value.cpp}
\subsection{Manacher}
\lstinputlisting{../codes/String/Manacher.cpp}
\subsection{Lexicographically Smallest Rotation}
\lstinputlisting{../codes/String/smallest_rotation.cpp}
\subsection{BWT}
\lstinputlisting{../codes/String/BWT.cpp}
\subsection{Palindromic Tree}
\lstinputlisting{../codes/String/palindromic_tree.cpp}

\section{Misc}
\subsection{Theorems}
\subsubsection{Sherman-Morrison formula}
$\left(A + uv^\textsf{T}\right)^{-1} = A^{-1} - \frac{A^{-1}uv^\textsf{T}A^{-1}}{1 + v^\textsf{T}A^{-1}u}$

\subsubsection{Kirchhoff's Theorem}
Denote $L$ be a $n \times n$ matrix as the Laplacian matrix of graph $G$, where $L_{ii} = d(i)$, $L_{ij} = -c$ where $c$ is the number of edge $(i, j)$ in $G$.
\begin{itemize}
    \item The number of undirected spanning in $G$ is $\lvert \det(\tilde{L}_{11}) \rvert$.
    \item The number of directed spanning tree rooted at $r$ in $G$ is $\lvert \det(\tilde{L}_{rr}) \rvert$.
\end{itemize}

\subsubsection{Tutte's Matrix}
Let $D$ be a $n \times n$ matrix, where $d_{ij} = x_{ij}$ ($x_{ij}$ is chosen uniform randomly) if $i < j$ and $(i, j) \in E$, otherwise $d_{ij} = -d_{ji}$. $\frac{rank(D)}{2}$ is the maximum matching on $G$.

\subsubsection{Cayley's Formula}
\begin{itemize}
  \item Given a degree sequence $d_1, d_2, \ldots, d_n$ for each labeled vertices, there're $\frac{(n - 2)!}{(d_1 - 1)!(d_2 - 1)!\cdots(d_n - 1)!}$ spanning trees.
  \item Let $T_{n, k}$ be the number of labeled forests on $n$ vertices with $k$ components, such that vertex $1, 2, \ldots, k$ belong to different components. Then $T_{n, k} = kn^{n - k - 1}$.
\end{itemize}

\subsubsection{Erdős–Gallai theorem}
A sequence of non-negative integers $d_1 \geq d_2 \geq \ldots \geq d_n$ can be represented as the degree sequence of a finite simple graph on $n$ vertices if and only if $d_1 + d_2 + \ldots + d_n$ is even and
$$ \sum_{i = 1}^{k}d_i \leq k(k - 1) + \sum_{i = k + 1}^{n}\min(d_i, k) $$
holds for all $1 \leq k \leq n$.

\subsubsection{Havel–Hakimi algorithm}
find the vertex who has greatest degree unused, connect it with other greatest vertex.

%\subsubsection{Hall's marriage theorem}
%Let $G$ be a finite bipartite graph with bipartite sets $X$ and $Y$.
%For a subset $W$ of $X$, let $N_G(W)$ denote the set of all vertices in $Y$ adjacent to some element of $W$. Then there is an $X$-saturating matching iff $\forall W \subseteq X, |W| \leq |N_G(W)|$

\subsubsection{Euler's planar graph formula}
$V-E+F=C+1$. $E\leq 3V-6$ (when $V\geq 3$)

\subsubsection{Pick's theorem}
For simple polygon, when points are all integer, we have $A=\text{\#\{lattice points in the interior\}} + \frac{\text{\#\{lattice points on the boundary\}}}{2} - 1$

%\subsubsection{Lucas's theorem}
%$\binom{m}{n}\equiv\prod_{i=0}^k\binom{m_i}{n_i}\pmod p$, where $m=m_kp^k+m_{k-1}p^{k-1}+\cdots +m_1p+m_0$, and $n=n_kp^k+n_{k-1}p^{k-1}+\cdots +n_1p+n_0$.

\subsubsection{Matroid Intersection}
Given matroids \(M_1=(G,I_1),M_2=(G,I_2)\), find maximum \(S\in I_1\cap I_2\).
For each iteration, build the directed graph and find a shortest path from \(s\) to \(t\).
\begin{itemize}
    \item \(s \to x: S \sqcup \{x\} \in I_1\)
    \item \(x \to t: S \sqcup \{x\} \in I_2\)
    \item \(y \to x: S \setminus \{y\} \sqcup \{x\} \in I_1\) (\(y\) is in the unique circuit of \(S \sqcup \{x\}\))
    \item \(x \to y: S \setminus \{y\} \sqcup \{x\} \in I_2\) (\(y\) is in the unique circuit of \(S \sqcup \{x\}\))
\end{itemize}
Alternate the path, and \(|S|\) will increase by \(1\).
Let \(R = \min(\operatorname{rank}(I_1), \operatorname{rank}(I_2)), N = |G|\).
In each iteration, \(|E| = O(RN)\).
For weighted case, assign weight \(-w(x)\) and \(w(x)\) to \(x\in S\) and \(x\notin S\), resp.
Use Bellman-Ford to find the weighted shortest path.
The maximum iteration of Bellman-Ford is \(2R+1\).

\subsection{MaximumEmptyRect}
\lstinputlisting{../codes/Misc/EmptyRect.cpp}
\subsection{DP-opt Condition}
\subsubsection{totally monotone (concave/convex)}
$\forall i < i^\prime, j < j^\prime$, $B[i][j] \leq B[i^\prime][j] \implies B[i][j^\prime] \leq B[i^\prime][j^\prime]$ \\
$\forall i < i^\prime, j < j^\prime$, $B[i][j] \geq B[i^\prime][j] \implies B[i][j^\prime] \geq B[i^\prime][j^\prime]$
\subsubsection{monge condition (concave/convex)}
$\forall i < i^\prime, j < j^\prime$, $B[i][j] + B[i^\prime][j^\prime] \geq B[i][j^\prime] + B[i^\prime][j]$ \\
$\forall i < i^\prime, j < j^\prime$, $B[i][j] + B[i^\prime][j^\prime] \leq B[i][j^\prime] + B[i^\prime][j]$

\subsection{Convex 1D/1D DP}
\lstinputlisting{../codes/Misc/Convex1D1D.cpp}
\subsection{Josephus Problem}
\lstinputlisting{../codes/Misc/Josephus.cpp}
\subsection{Cactus Matching}
\lstinputlisting{../codes/Misc/CactusMatching.cpp}
\subsection{DLX}
\lstinputlisting{../codes/Misc/DLX.cpp}

\end{document}
