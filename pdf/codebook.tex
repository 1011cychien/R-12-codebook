\documentclass[a4paper,10pt,twocolumn,oneside]{article}
\setlength{\columnsep}{10pt}                                                                    %兩欄模式的間距
\setlength{\columnseprule}{0pt}                                                                %兩欄模式間格線粗細

\usepackage{amsthm}								%定義,例題
\usepackage{amssymb}
%\usepackage[margin=2cm]{geometry}
\usepackage{fontspec}								%設定字體
\usepackage{color}
\usepackage[x11names]{xcolor}
\usepackage{xeCJK}								%xeCJK
\usepackage{listings}								%顯示code用的
%\usepackage[Glenn]{fncychap}						%排版,頁面模板
\usepackage{fancyhdr}								%設定頁首頁尾
\usepackage{graphicx}								%Graphic
\usepackage{enumerate}
\usepackage{titlesec}
\usepackage{amsmath}
\usepackage{pdfpages}
%\usepackage[T1]{fontenc}
\usepackage{amsmath, courier, listings, fancyhdr, graphicx}
\topmargin=0pt
\headsep=5pt
\textheight=780pt
\footskip=0pt
\voffset=-40pt
\textwidth=545pt
\marginparsep=0pt
\marginparwidth=0pt
\marginparpush=0pt
\oddsidemargin=0pt
\evensidemargin=0pt
\hoffset=-42pt

%\renewcommand\listfigurename{圖目錄}
%\renewcommand\listtablename{表目錄} 

%%%%%%%%%%%%%%%%%%%%%%%%%%%%%

%\setmainfont{Consolas}				%主要字型
\setmainfont{Helvetica}
%\setmainfont{Linux Libertine G}
\setmonofont{Courier New}
%\setmonofont{Source Code Pro}
%\setCJKmainfont{Source Han Sans}			%中文字型
\setCJKmainfont{PingFang TC}
%\setmainfont{sourcecodepro}
\XeTeXlinebreaklocale "zh"						%中文自動換行
\XeTeXlinebreakskip = 0pt plus 1pt				%設定段落之間的距離
\setcounter{secnumdepth}{3}						%目錄顯示第三層

%%%%%%%%%%%%%%%%%%%%%%%%%%%%%
\makeatletter
\lst@CCPutMacro\lst@ProcessOther {"2D}{\lst@ttfamily{-{}}{-{}}}
\@empty\z@\@empty
\makeatother
\lstset{											% Code顯示
language=C++,										% the language of the code
basicstyle=\footnotesize\ttfamily, 						% the size of the fonts that are used for the code
%numbers=left,										% where to put the line-numbers
numberstyle=\footnotesize,						% the size of the fonts that are used for the line-numbers
stepnumber=1,										% the step between two line-numbers. If it's 1, each line  will be numbered
numbersep=5pt,										% how far the line-numbers are from the code
backgroundcolor=\color{white},					% choose the background color. You must add \usepackage{color}
showspaces=false,									% show spaces adding particular underscores
showstringspaces=false,							% underline spaces within strings
showtabs=false,									% show tabs within strings adding particular underscores
frame=false,											% adds a frame around the code
tabsize=2,											% sets default tabsize to 2 spaces
captionpos=b,										% sets the caption-position to bottom
breaklines=true,									% sets automatic line breaking
breakatwhitespace=false,							% sets if automatic breaks should only happen at whitespace
escapeinside={\%*}{*)},							% if you want to add a comment within your code
morekeywords={*},									% if you want to add more keywords to the set
keywordstyle=\bfseries\color{Blue1},
commentstyle=\itshape\color{Red4},
stringstyle=\itshape\color{Green4},
}

%%%%%%%%%%%%%%%%%%%%%%%%%%%%%

\def\footnotesize{\fontsize{8}{9}\selectfont}

\begin{document}
\pagestyle{fancy}
\fancyfoot{}
%\fancyfoot[R]{\includegraphics[width=20pt]{ironwood.jpg}}
\fancyhead[L]{oToToT Will Go To IOI2018}
\fancyhead[R]{\thepage}
\renewcommand{\headrulewidth}{0.4pt}
\renewcommand{\contentsname}{Contents} 

\scriptsize
\tableofcontents
%%%%%%%%%%%%%%%%%%%%%%%%%%%%%

\newpage

\section{Basic}
\subsection{Default Code}
\lstinputlisting{../codes/Basic/default.cpp}

\subsection{IncreaseStackSize}
\lstinputlisting{../codes/Basic/IncStack.cpp}

\subsection{Pragma optimization}
\lstinputlisting{../codes/Basic/Pragma.cpp}

\subsection{Debugger}
\lstinputlisting{../codes/Basic/debug_checker.py}

\subsection{Quick Random}
\lstinputlisting{../codes/Basic/random_generator.cpp}

\subsection{IO Optimization}
\lstinputlisting{../codes/Basic/IOOptimize.cpp}

\section{Data Structure}
\subsection{Bigint}
\lstinputlisting{../codes/Data_Structure/Bigint.cpp}

\subsection{ScientificNotation}
\lstinputlisting{../codes/Data_Structure/SciFi.cpp}

\subsection{unordered\_map}
\lstinputlisting{../codes/Data_Structure/hash_table.cpp}

\subsection{extc\_balance\_tree}
\lstinputlisting{../codes/Data_Structure/ExtTree.cpp}

\subsection{extc\_heap}
\lstinputlisting{../codes/Data_Structure/ExtHeap.cpp}

\subsection{SkewHeap}
\lstinputlisting{../codes/Data_Structure/SkewHeap.cpp}

\subsection{Disjoint Set}
\lstinputlisting{../codes/Data_Structure/DisjointSet.cpp}

\subsection{Treap}
\lstinputlisting{../codes/Data_Structure/Treap.cpp}

\subsection{SparseTable}
\lstinputlisting{../codes/Data_Structure/SparseTable.cpp}

\subsection{FenwickTree}
\lstinputlisting{../codes/Data_Structure/FenwickTree.cpp}

\section{Graph}

\subsection{BCC Edge}
\lstinputlisting{../codes/Graph/BCC_Edge.cpp}

\subsection{BCC Vertex}
\lstinputlisting{../codes/Graph/BCC_Vertex.cpp}

\subsection{Strongly Connected Components}
\lstinputlisting{../codes/Graph/Kosaraju.cpp}

\subsection{Bipartie Matching}
\lstinputlisting{../codes/Graph/BI_Pair.cpp}

\subsection{MinimumCostMaximumFlow}
\lstinputlisting{../codes/Graph/CostFlow.cpp}

\subsection{MaximumFlow}
\lstinputlisting{../codes/Graph/Dinic.cpp}

\subsection{Kuhn Munkres}
\lstinputlisting{../codes/Graph/KM.cpp}

\section{Math}
\subsection{Prime Table}
\lstinputlisting{../codes/Math/primes.txt}

\subsection{ax+by=gcd}
\lstinputlisting{../codes/Math/ExtendedGCD.cpp}

\subsection{Pollard Rho}
\lstinputlisting{../codes/Math/PollardRho.cpp}

\subsection{Linear Sieve}
\lstinputlisting{../codes/Math/LinearSieve.cpp}

\subsection{NloglogN Sieve}
\lstinputlisting{../codes/Math/QuickSieve.cpp}

\subsection{Range Sieve}
\lstinputlisting{../codes/Math/RangeSieve.cpp}

\subsection{Miller Rabin}
\lstinputlisting{../codes/Math/miller_rabin.cpp}

\subsection{Inverse Element}
\lstinputlisting{../codes/Math/Inverse.cpp}

\subsection{Euler Phi Function}
\lstinputlisting{../codes/Math/euler_phi.cpp}

\subsection{Gauss Elimination}
\lstinputlisting{../codes/Math/GaussElimination.cpp}

\subsection{Fast Fourier Transform}
\lstinputlisting{../codes/Math/FFT.cpp}

\subsection{Chinese Remainder}
\lstinputlisting{../codes/Math/ChineseRemainder.cpp}

%\subsection{Fast Linear Recurrence}
%\lstinputlisting{../codes/Math/Fast_Linear_Recurrence.cpp}

\subsection{NTT}
\lstinputlisting{../codes/Math/NTT_eddy.cpp}

%\subsection{(+1) PolynomialGenerator}
%\lstinputlisting{../codes/Math/PolyGen.cpp}

%\subsection{Pseudoinverse of Square matrix}
%\lstinputlisting{../codes/Math/Square_Matrix_pinv.cpp}

%\newpage

\section{Geometry}
\subsection{Point Class}
\lstinputlisting{../codes/Geometry/point_class.cpp}

\subsection{Circle Class}
%\input{../codes/Geometry/Intersection_of_two_circles/Intersection_of_two_circles.tex}
\lstinputlisting{../codes/Geometry/Circle.cpp}

\subsection{Line Class}
\lstinputlisting{../codes/Geometry/line_class.cpp}

\subsection{Segment Class}
\lstinputlisting{../codes/Geometry/segment_class.cpp}

\subsection{Triangle Circumcentre}
\lstinputlisting{../codes/Geometry/outer_point.cpp}

%TODO wrong when intersection is empty
%\subsection{Half Plane Intersection}
%\lstinputlisting{../codes/Geometry/Half_plane_intersection/half_plane_intersection.cpp}

\subsection{2D Convex Hull}
\lstinputlisting{../codes/Geometry/2D_ConvexHull.cpp}

\subsection{SimulateAnnealing}
\lstinputlisting{../codes/Geometry/SimulateAnnealing.cpp}

%\subsection{3D Convex Hull}
%\lstinputlisting{../codes/Geometry/Convex_Hull/3D_convex_hull.cpp}

\subsection{Minimum Covering Circle}
\lstinputlisting{../codes/Geometry/MinCircleCover.cpp}

%\subsection{KDTree (Nearest Point)}
%\lstinputlisting{../codes/Geometry/KD_tree/KD_Tree.cpp}

%\subsection{Triangulation}
%\lstinputlisting{../codes/Geometry/triangulation/triangulation.cpp}

%\subsection{(+1) MinkowskiSum}
%\lstinputlisting{../codes/Geometry/Minkowski_Sum/Minkowski_Sum.cpp}

\section{Stringology}

\subsection{Hash}
\lstinputlisting{../codes/String/Hash.cpp}

\subsection{Suffix Array}
\lstinputlisting{../codes/String/Suffix_Array.cpp}

%\subsection{Aho-Corasick Algorithm}
%\lstinputlisting{../codes/Stringology/Automata/Aho-Corasick.cpp}

\subsection{KMP}
\lstinputlisting{../codes/String/KMP.cpp}

\subsection{Z value}
\lstinputlisting{../codes/String/Z_Value.cpp}

%\subsection{Z value (palindrome ver.)}
%\lstinputlisting{../codes/Stringology/Z_Value/zvalue_palindrome.cpp}

%\subsection{palindromic tree}
%\lstinputlisting{../codes/Stringology/palindromic_tree/palindromic_tree.cpp}

\subsection{Lexicographically Smallest Rotation}
\lstinputlisting{../codes/String/smallest_rotation.cpp}

%\subsection{Suffix Automaton}
%\lstinputlisting{../codes/Stringology/Automata/SAM.cpp}

\end{document}
